\chapter{OpenAI Gym}

\textit{OpenAI Gym} je \textit{Python} biblioteka \engl{library} otvorenog koda \engl{open source} koja služi za razvijanje i usporedbu agenata u odabranim okolinama. Iznimno je popularna u sferi simpatizera i programera koji se bave razvijanjem modela podržanog učenja zbog jednostavnosti korištenja, velikog broja dostupnih okolina i jednostavnog stvaranja novih okolina, te jednostavne interakcije agenta i okoline. \textit{OpenAI Gym} biblioteka se redovito održava i trenutno je u verziji \texttt{0.24.1}. 

\section{Struktura}

Interakcija agenta i okoline podijeljena je na epizode. Na početku svake epizode, početno stanje se nasumično uzorkuje iz distribucije, i interakcija se nastavlja sve dok se okolina nađe u terminalnom stanju \cite{OpenAIWhitepaper}.

\begin{listing}[H]
    \caption{Jednostavan primjer integracije agenta i \textit{Gym} okoline (1 epizoda)}
    \inputminted{python}{snippets/init.py}
    \label{lst:init-code}
\end{listing}

Kod \ref{lst:init-code} prikazuje potpunu implementaciju jednostavne interakcije agenta i okoline. Agent u ovom jednostavnom slučaju nasumično odabere akciju iz skupa svih dostupnih akcija za tu okolinu (linija 9). Osnovni kostur sastoji se od koraka specifikacije okoline (linija 3), inicijalizacije okoline (linija 5) te interakcija okoline i agenta - agent predaje okolini odabranu akciju, okolina vraća povratnu informaciju (linije 7 - 14). 

\begin{figure}[h]
    \centering
    \frame{\includegraphics[width=10cm]{assets/mountain-car.png}}
    \caption{Rezultat pokretanja koda \ref{lst:init-code}}
    \label{fig:mountain-car}
\end{figure}

\subsection{Okolina}

Temelj oko kojeg se zasniva \textit{OpenAI Gym} biblioteka jest razred \engl{class} \texttt{Env} koji u suštini implementira simulator koji pokreće okruženje u kojem naš agent može interaktirati s okolinom. Točnije rečeno, enkapsulira sva potrebna ponašanja i metode koje su potrebne za jednostavnu interakciju. Objekt tipa \texttt{Env} stvara se pozivanjem funkcije \texttt{gym.make()} (kod \ref{lst:init-code} linija 3) kojoj se predaje identifikator okoline (\texttt{id}) zajedno s opcionalnim argumentima (metapodatcima). 

\subsection{Interakcija s okolinom}

Kao što je vidljivo iz koda \ref{lst:init-code} osnovne metode koje se pozivaju nad instancom razreda \texttt{Env} su \texttt{reset} i \texttt{step}. Funkcija \texttt{reset} postavlja okruženje u početno stanje i vraća njegovu vrijednost (kod \ref{lst:init-code}. linija 5). S druge strane, funkciji \texttt{step} (kod \ref{lst:init-code}. linija 10) predaje se jedna od ispravnih akcija koja inicira prijelaz okoline iz jednog stanja u drugo. Funkcija vraća 4 vrijednosti: vrijednost prostora stanja \engl{observation}, iznos nagrade kao rezultat poduzimanja određene akcije  \engl{reward}, zastavicu koja signalizira jesmo li došli u završno stanje okoline \engl{done}, te neke dodatne informacije.

Još jedna vrlo često korištena funkcija jest \texttt{render} (kod \ref{lst:init-code}. linija 12) koja služi kako bi se u određenom formatu prikazala okolina. Dostupni formati su: \texttt{human} (otvara se skočni prozor sa slikom stanja okoline - slika \ref{fig:mountain-car}), \texttt{rgb_array} (\texttt{numpy} array RGB vrijednosti) i \texttt{ansi} (string reprezentacija okoline).

\subsection{Prostor akcija i prostor stanja}

Osnovna struktura okruženja opisana je atributima \texttt{observation_space} i \texttt{actio\-n_space} koji su dio razreda \texttt{Env} i čija se vrijednost može razlikovati zavisno o okolini. Atribut \texttt{action_space} opisuje numeričku strukturu svih legitimnih akcija koje se mogu izvesti nad određenom okolinom. S druge strane, atribut \texttt{observation_s\-pace} definira strukturu objekta koje predstavlja opis stanja u kojem se okolina nalazi.

Format validnih akcija i stanja okoline, odnosno struktura tih podataka, definirana je razredima \texttt{Box}, \texttt{Discrete}, \texttt{MultiBinary} i \texttt{MultiDiscrete}. Svi navedeni razredi nasljeđuju i implementiraju glavne metode nadrazreda \texttt{Space}. 

Razred \texttt{Box} predstavlja strukturu podataka u kontinuiranom $n$-dimenzionalnom prostoru. Prostor i njegove validne vrijednosti omeđene su gornjim i donjim granicama koje se jednostavno postave pri inicijalizaciji strukture pridruživanjem željenih vrijednosti atributima \texttt{high} i \texttt{low}. Kod \ref{lst:box-code} prikazuje inicijalizaciju \texttt{Box} strukture podataka koja je sastavljena od $3$-dimenzionalnog vektora čije su vrijednosti omeđene odozdo i odozgo vrijednostima $-1$ i $-2$. Metoda \texttt{sample(self)} nasumično uzorkuje element iz prostora koristeći različite distribucije ovisno o ograničenjima prostora.

\begin{listing}[H]
    \caption{Primjer korištenja strukture kontinuiranog prostora \texttt{Box}}
    \inputminted{python}{snippets/box.txt}
    \label{lst:box-code}
\end{listing}

Razred \texttt{Discrete} s druge strane, predstavlja strukturu podataka u diskretnom $n$-dimenzionalnom prostoru gdje su validne vrijednosti sve cjelobrojne vrijednosti unutar intervala $[0, n-1]$ (početna vrijednost se može specificirati). Kod \ref{lst:discrete-code} prikazuje inicijalizacije \texttt{Discrete} strukture podataka ovisno o specificiranoj početnoj vrijednosti.

\begin{listing}[H]
    \caption{Primjer korištenja strukture diskretnog prostora \texttt{Discrete}}
    \inputminted{python}{snippets/discrete.txt}
    \label{lst:discrete-code}
\end{listing}

\subsection{Omotači}

Omotači \engl{Wrappers} su prikladne strukture koje omogućavaju izmjenu elemenata postojećeg okruženja bez potrebe za mijenjanjem originalnog koda. Omotači omogućavaju modularnost, mogu se implementirati prema vlastitim potrebama i ulančavati. Ova funkcionalnost vrlo se često koristi u situacijama kada pri treniranju modela želimo normalizirati ulaze (skalirati vrijednosti slikovnih jedinica), provesti regularizaciju (podrezivanje vrijednosti nagrade), transformirati ulaze u \textit{PyTorch} dimenzije, implementirati metodu preskakanja slikovnih okvira...  Navedene funkcionalnosti moguće je postići tako da definiramo vlastiti omotač koji će nasljeđivati ili obični \texttt{Wrapper} nadrazred ili specifičnije razrede poput \texttt{ObservationWrapper}, \texttt{RewardWrapper}, \texttt{ActionWrapper}...

\subsection{Vektorizirana okruženja}

Koristeći standardne metode stvaranja i interakcije s Gym okruženjem, pokrećemo samo jednu instancu okruženja i na taj način ne iskorištavamo računalnu snagu koja nam je dostupna u potpunosti. Vektorizirana okruženja \engl{Vectorized environments} su okruženja koja paralelno pokreću više kopija istog okruženja u svrhu poboljšanja učinkovitosti i ubrzanja procesa učenja agenta i njegove interakcije s kopijama okolina.

\textit{Python} biblioteka \textit{stable-baselines} \cite{BaselinesVecEnvs} pruža gotove omotače koji omogućavaju pokretanje $n$ paralelnih neovisnih instanci okolina. Paralelna obrada interno je implementirana korištenjem \textit{multiprocessing} paketa. Zbog paralelne interakcije agenta s više okolina potrebno je prilagoditi strukturu objekata. Agent predaje omotanoj okolini niz od $n$ akcija i dobiva listu povratnih informacija (stanje pojedinačne okoline, nagrada i zastavica terminalnog stanja) u formi liste $n$ vrijednosti.

\begin{figure}[H]
    \centering
    \frame{\includegraphics[height=9cm]{assets/breakout-vect.png}}
    \caption{Vektorizirano okruženje sa 4 paralelne asinkrone instance}
    \label{fig:breakout-vect}
\end{figure}

\section{Determinizam atari okruženja}

Upravljanje agentima i njihovo izvođenje akcija u određenoj atari okolini interno je implementirano koristeći objektno orijentiranu razvojnu cjelinu \engl{framework} \textit{The Arcade Learning Environment} (skraćeno \textit{ALE}). Na taj način razdvajaju se slojevi emulacije okruženja (koristeći Atari 2600 emulator \textit{Stella}), sloj koji kontrolira samog agenta \textit{(ALE)} i sloj koji pruža jednostavnu programsku implementaciju i interakciju \textit{(OpenAI Gym)} \cite{OpenAIALE}. 

Originalna Atari 2600 konzola nije imala izvor entropije za generiranje pseudoslučajnih brojeva i iz tog razloga okruženje je bilo u potpunosti determinističko - svaka igra počinje u istom stanju, a ishodi su u potpunosti određeni stanjem i akcijom \cite{AleDeterministic}. Iz tog razloga vrlo je jednostavno postići visoku uspješnost agenta u okolini pamćenjem dobrog niza akcija. Nama to ne odgovara jer želimo postići da agent nauči donositi dobre odluke. Pristupi kojima se nastoji uvesti određen stupanj stohastike su \textit{sticky actions}, \textit{frame skipping}, \textit{initial no-ops}, te \textit{random action noise}. 

\textit{Frame skipping} pristup pri svakom koraku okoline uzima zadnju akciju koju je agent odabrao i ponavlja ju $n$ broj puta (kroz $n$ slikovnih okvira). Ovisno o verzijama okoline, broj ponavljanja može se specificirati tipom \texttt{int} - cijelobrojnom fiksnom vrijednošću (determinizam), ili tipom \texttt{tuple(2)} - bira se slučajno odabrana vrijednost unutar specificiranog intervala (stohastika). Osim potencijalnog uvođenja stohastike, ovaj pristup pojednostavljuje problem podržanog učenja i ubrzava izvođenje \cite{AleDeterministic}. Tehnika \textit{initial no-ops} označava da na početku epizode okolina ignorira akcije agenta slučajan broj puta iz intervala $[0, n]$ - izvršava akciju \texttt{NOOP}. Osim što na početku agent ne zna kada će okolina početi koristiti njegove akcije, ostatak interakcije s okolinom je i dalje deterministički \cite{AleDeterministic}. Sličan pristup bio bi i da potičemo raznolikost početnih stanja objekta s kojim upravljamo. To bi postigli nasumičnim poduzimanjem niza akcija kojima pomičemo objekt prije samog aktiviranja posebne akcije \texttt{FIRE} \cite{MediumDeterministic}.

Tehnika \textit{random action noise} zamjenjuje odabranu akciju agenta nasumičnom akcijom samo ako je vjerojatnost zamjene manja od specificirane vrijednosti. Ovo uspješno uvođenje stohastike dolazi s negativnim posljedicama poduzimanja slučajne akcije koja može značajno poremetiti politiku agenta i smanjiti učinak \cite{AleDeterministic}. Odabir slučajne akcije bolje je zamijeniti pristupom ponavljanja prethodno odabrane akcije uzimajući u obzir parametar vjerojatnosti zamjene akcije - \textit{stickiness}. Riječ je o \textit{sticky actions} pristupu koji uvodi stohastiku, te ne ometa odabir akcije agenta koji može biti siguran da njegove akcije neće biti pogubne za ostatak iterakcije s okolinom.

Pristupi \textit{sticky actions} i \textit{frame skipping} dio su \textit{OpenAI Gym} biblioteke i mogu se specificirati prilikom instanciranja okoline, dok je \textit{random action noise} i \textit{initial no-ops} tehnike potrebno manualno implementirati ili se osloniti na izvedbu u poznatim \textit{Python} bibliotekama poput \textit{stable-baselines3}.

\section{Okruženja}

U OpenAI Gym ekosustavu dostupno je puno okruženja koja omogućuju interakciju s agentom. Neki od njih su: \textit{Atari} - skup od Atari 2600 okolina, \textit{MuJoCo} (punim nazivom \textit{Multi-Joint dynamics with Contact}) - skup okolina za provođenje istraživanja i razvoja u robotici, biomehanici, grafici i drugim područjima gdje je potrebna brzina i točna simulacija, \textit{Classic Control} - skup okolina koje opisuju poznate fizikalne eksperimente. Opisat ćemo neke od najkorištenijih okolina.

\subsection{Okruženje CartPole}

Ovo okruženje modelira fizikalni problem održavanja ravnoteže. Inačica je sličnog fizikalnog problema pod nazivom \textit{obrnuto njihalo} \engl{inverted pendulum}. Za pomična kolica zakvačen je stupić. Njegovo težište nalazi se iznad središta mase i na taj način osigurava da sustav nije stabilan.  Zglob, odnosno dodirna točka između stupića i kolica nema trenja niti drugih gubitaka. Također, kolica koja se kreću vodoravno po putanji u 2 smjera nemaju trenja niti drugih gubitaka. Cilj ovog fizikalnog problema jest uravnotežiti stup primjenom sila i pomicanjem kolica u lijevom ili desnom smjeru.

Za svaki poduzeti korak okolina dodjeljuje nagradu u vrijednosti $+1$. Struktura valjanih akcija koje agent može poduzeti (\texttt{action_space}) instanca je razreda \texttt{Disc\-rete(2)} - skup akcija je diskretan i u svakom koraku je moguće odabrati 1 od maksimalno 2 dostupne akcije. Opis značenja svake akcije prikazan je u tablici \ref{table:cart-pole-action}. S druge strane, objekt koji predstavlja strukturu stanja okoline u određenom vremenskom trenutku (\texttt{observation_space}) instanca je razreda \texttt{Box(4)} – stanje se sastoji od 4 kontinuirane vrijednosti od kojih su neke ograničene i odozdo i odozgo. Točan opis i granice vrijednosti predočeni su u tablici \ref{table:cart-pole-observation}.

\begin{table}[ht]
    \centering
    \caption{Opis valjanih akcija okoline CartPole - atribut \texttt{action_space}}
    \begin{tabular}{c c}
        \toprule
        Akcija & Opis akcije  \\
        \midrule
        0 & Pomak kolica ulijevo \\
        1 & Pomak kolica udesno \\
        \bottomrule
    \end{tabular}
    \label{table:cart-pole-action}
\end{table}

\begin{table}[ht]
    \centering
    \caption{Opis strukture okoline okoline CartPole - atribut \texttt{observation_space}}
    \begin{tabular}{c c c c}
        \toprule
        Indeks & Opis & Donja granica & Gornja granica \\
        \midrule
        0 & Pozicija kolica & $-4.8$ & $4.8$ \\
        1 & Brzina kolica & $-\infty$ & $\infty$ \\ 
        2 & Nagib štapića i kolica & $-0.418 \text{rad}$ & $0.418 \text{rad}$ \\
        3 & Brzina štapića na vrhu & $-\infty$ & $\infty$ \\
        \bottomrule
    \end{tabular}
    \label{table:cart-pole-observation}
\end{table}

Početno stanje okoline inicijalizira se pozivom metode \texttt{reset()} slučajnim vrijednostima iz uniformne razdiobe na intervalu $[- 0.05, 0.05]$. Okolina podržava 3 uvjeta zaustavljanja (uvjeti koji označuju da je riječ o terminalnom stanju): nagib štapića i kolica je izvan intervala $[-0.2095, 0.2095] \text{rad}$, pozicija sredine kolica je izvan intervala $[-2.4, 2.4]$ (sredina kolica dotiče rub vidljivog prostora) i duljina epizode veća je od $500$ koraka. Na slici \ref{fig:cart-pole} prikazana je okolina CartPole zajedno s vrijednostima okoline.

\begin{figure}[H]
    \centering
    \frame{\includegraphics[height=5.5cm]{assets/cart-pole-not-mine.png}}
    \caption{Izgled i vrijednosti CartPole okoline \cite{CartPoleValues}}
    \label{fig:cart-pole}
\end{figure}

\subsection{Okruženje Breakout}

Ova okolina simulira poznatu Atari 2600 igru u kojoj je cilj sakupiti što više bodova pomičući platformu i održavajući lopticu na ekranu. Platforma je postavljena na dnu ekrana, na fiksnoj visini i moguće ju je pomicati u dva smjera. Loptica se odbija između zidova, platforme i 6 razina \textit{ciglenih} blokova čijim razbijanjem se sakupljaju bodovi. Ako loptica padne ispod platforme koju igrač kontrolira, gubi se život. Igra završava kada igrač potroši 5 života, odnosno kada 5 puta loptica padne ispod platforme. 

Skup validnih akcija je instanca razreda \texttt{Discrete(4)} - skup akcija je diskretan i u svakom koraku je moguće odabrati 1 od maksimalno 4 dostupne akcije. Opis značenja svake akcije prikazan je u tablici \ref{table:breakout-action}. Kao opis trenutnog stanja okoline moguće je dobiti RGB vrijednosti svakog piksela slike (slike dimenzije $210 \times 160$) ili vrijednosti radne memorije \textit{ALE} okoline (128 bajta) - što je korisno jer možemo preskočiti korak učenja reprezentacije okoline (preskačemo dio gdje algoritmi učenja moraju iz piksela slike naučiti reprezentaciju). Razlike u strukturi objekta okoline prikazane su u tablici \ref{table:breakout-observation}. 

\begin{table}[ht]
    \centering
    \caption{Opis valjanih akcija okoline Breakout - atribut \texttt{action_space}}
    \begin{tabular}{c c c}
        \toprule
        Akcija & Opis akcije & Detaljniji opis akcije  \\
        \midrule
        0 & NOOP & Ne poduzima se nikakva akcija \\
        1 & FIRE & Akcija koja pokreće igru \\ 
        2 & RIGHT & Platforma se pomiče udesno \\ 
        3 & LEFT & Platforma se pomiče ulijevo  \\
        \bottomrule
    \end{tabular}
    \label{table:breakout-action}
\end{table}

\begin{table}[H]
    \centering
    \caption{Opis strukture objekta okoline Breakout - atribut \texttt{observation_space}}
    \begin{tabular}{c c}
        \toprule
        Indeks & Struktura  \\
        \midrule
        RAM vrijednosti & \texttt{Box(0, 255, (128,), uint8)}  \\ 
        Vrijednosti RGB slike & \texttt{Box(0, 255, (210, 160, 3), uint8)}  \\
        \bottomrule
    \end{tabular}
    \label{table:breakout-observation}
\end{table}

Nagrada dolazi u obliku bodova koji se dobivaju uništavajući \textit{ciglene} blokove. Vrijednost nagrade ovisi o boji cigle. Izgled same okoline prikazan je na slici \ref{fig:breakout}.

\begin{figure}[H]
    \centering
    \frame{\includegraphics[width=10cm]{assets/breakout.png}}
    \caption{Izgled Breakout okoline}
    \label{fig:breakout}
\end{figure}


% TODO dodati za literaturu https://blog.paperspace.com/getting-started-with-openai-gym/ i https://www.gymlibrary.ml/content/vector_api/ i https://www.gymlibrary.ml/ općenito

% TODO dodati za literaturu https://atariage.com/manual_html_page.php?SoftwareID=889

% značenje pojedinih ram pozicija https://github.com/mila-iqia/atari-representation-learning/blob/master/atariari/benchmark/ram_annotations.py

% https://stackoverflow.com/questions/45207569/how-to-interpret-the-observations-of-ram-environments-in-openai-gym