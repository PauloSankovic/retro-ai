\chapter{Uvod}
Inteligentni sustavi sposobni su analizirati, razumjeti i učiti iz dostupnih podataka putem posebno dizajniranih algoritama umjetne inteligencije. Zahvaljujući dostupnosti velikih skupova podataka, razvoja tehnologije i napretka u algoritmima došli smo do razine gdje nam razvijeni modeli mogu uvelike poslužiti u svakodnevnom životu.

Posebno je zanimljiva problematika u kojoj je bez znanja o pravilima i funkcioniranju specifične okoline, potrebno konstruirati specijaliziranog agenta koji se nalazi u određenom stanju okoline i ponavlja korake izvršavanja optimalne akcije i prijelaza u novo stanje okoline. Za svaku akciju agent prima određenu nagradu - mjeru koja označava koliko su akcije agenta ispravne za tu okolinu i koliko je napredak agenta ispravan. Agent izvršava akcije i prelazi u nova stanja sve dok se ne nađe u terminalnom (završnom) stanju. Dakle, cilj agenta u okolini jest pronaći optimalnu strategiju koja će maksimizirati očekivanu dobit (nagradu) u određenom vremenskom okviru.

Područje strojnog učenja koje se bavi prethodno navedenom problematikom naziva se podržano učenje \engl{Reinforcement Learning}. Agenti koji se prilično dobro ponašaju u takvim okolinama možemo implementirati pomoću različitih algoritama podržanog učenja koji se temelje na umjetnim neuronskim mrežama \engl{Artificial Neural Networks}. Umjetne neuronske mreže su dobri aproksimatori funkcija i najbolje se ponašaju u okolini koja ima kompozitnu strukturu gdje vrlo kvalitetno duboki model predstave kao slijed naučenih nelinearnih transformacija.

U sklopu ovog rada bilo je potrebno proučiti i razumjeti metode i algoritme podržanog učenja i funkcioniranje umjetnih neuronskih mreža. Nadalje, bilo je potrebno istražiti, proučiti i naposljetku implementirati neke od algoritama podržanog učenja koji se zasnivaju na umjetnim neuronskim mrežama i koje je trebalo uklopiti u okoline koje su prikladne za simulaciju i testiranje ponašanja naučenih agenta.

Programsko rješenje implementirano je u programskom jeziku \textit{Python}, primarno koristeći \textit{PyTorch} biblioteku \engl{library} zajedno s ostalim korisnim bibliotekama poput \textit{numpy}, \textit{tqdm}, \textit{stable-baselines3}... Za simulaciju i testiranje ponašanja agenata (razvijenih modela) u posebnom okruženju korištena je biblioteka \textit{OpenAI Gym}. 
