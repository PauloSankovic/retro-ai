\begin{sazetak}

Inteligentni sustavi sposobni su analizirati, razumjeti i učiti iz dostupnih podataka putem posebno dizajniranih algoritama umjetne inteligencije. U ovom radu bavimo se problematikom u kojoj je bez znanja o pravilima i funkcioniranju specifične okoline potrebno konstruirati specijaliziranog agenta kojemu je cilj pronaći optimalnu strategiju koja će maksimizirati očekivanu dobit u određenom vremenskom okviru. Agenti koji se prilično dobro ponašaju u takvim okolinama možemo implementirati pomoću različitih algoritama podržanog učenja koji se temelje na dubokim modelima. 

U sklopu ovog rada objašnjena je osnovna ideja podržanog učenja, načini funkcioniranja dubokih unaprijednih potpuno povezanih i konvolucijskih modela, te arhitektura i ideja koja stoji iza odabranih algoritama podržanog učenja koji nisu ovisni o modelu. Detaljno je i prezentirana struktura \textit{OpenAi Gym} biblioteke i njenih okolina, te napravljena usporedba u kojoj se analizira uspješnost naučenih agenata.

Programsko rješenje implementirano je u programskom jeziku \textit{Python}, primarno koristeći \textit{PyTorch} biblioteku za dizajn i učenje dubokih neuronskih mreža, te \textit{OpenAI Gym} biblioteku za simulaciju i testiranje ponašanja agenata.


\kljucnerijeci{Podržano učenje, duboko učenje, neuronske mreže, OpenAI Gym, Model-Free algoritmi}
\end{sazetak}

\engtitle{Automated design of agents for various environments}
\begin{abstract}
Abstract.

\keywords{Reinforcement learning, deep learning, neural networks, OpenAI Gym, Model-Free RL algorithms}
\end{abstract}
